\documentclass[12pt,a4paper,dvipdfmx]{jarticle}

% pLaTeX用パッケージ
\usepackage[dvipdfmx]{graphicx}
\usepackage[dvipdfmx]{color}
\usepackage[dvipdfmx]{hyperref}

% その他のパッケージ
\usepackage{amsmath}
\usepackage{listings}

% タイトル情報
\title{pLaTeX サンプル文書}
\author{著者名}
\date{\today}

\begin{document}

\maketitle

\section{はじめに}

これはpLaTeXのサンプル文書です。日本語も問題なく表示されます。

\section{数式の例}

インライン数式の例: $E = mc^2$

ディスプレイ数式の例:
\begin{equation}
  \int_{-\infty}^{\infty} e^{-x^2} dx = \sqrt{\pi}
\end{equation}

複数行の数式:
\begin{align}
  f(x) &= x^2 + 2x + 1 \\
  &= (x + 1)^2
\end{align}

\section{リスト}

\subsection{箇条書き}

\begin{itemize}
  \item 項目1
  \item 項目2
  \item 項目3
\end{itemize}

\subsection{番号付きリスト}

\begin{enumerate}
  \item 最初の項目
  \item 二番目の項目
  \item 三番目の項目
\end{enumerate}

\section{コードの挿入}

\begin{lstlisting}[language=Python, caption=Pythonコード例]
def hello_world():
    print("Hello, World!")

if __name__ == "__main__":
    hello_world()
\end{lstlisting}

\section{画像の挿入}

\subsection{基本的な画像挿入}

画像を挿入する基本的な方法です:

\includegraphics{images/placeholder.png}

\subsection{サイズを指定した画像}

画像のサイズを指定することができます:

\includegraphics[width=0.5\textwidth]{images/placeholder.png}

\subsection{キャプション付き画像}

figure環境を使用すると、キャプションや参照ラベルを付けることができます。

\begin{figure}[htbp]
  \centering
  \includegraphics[width=0.6\textwidth]{images/placeholder.png}
  \caption{プレースホルダー画像の例}
  \label{fig:placeholder}
\end{figure}

図\ref{fig:placeholder}のように参照することができます。

\subsection{複数の画像を並べる}

複数の画像を横に並べることもできます:

\begin{figure}[htbp]
  \centering
  \begin{minipage}{0.45\textwidth}
    \centering
    \includegraphics[width=\textwidth]{images/placeholder.png}
    \caption{左の画像}
    \label{fig:left}
  \end{minipage}
  \hfill
  \begin{minipage}{0.45\textwidth}
    \centering
    \includegraphics[width=\textwidth]{images/placeholder.png}
    \caption{右の画像}
    \label{fig:right}
  \end{minipage}
\end{figure}

\section{まとめ}

この文書はpLaTeX環境のテストに使用できます。
\begin{itemize}
  \item pLaTeXでのコンパイル
  \item 日本語の表示
  \item 数式のレンダリング
  \item コードハイライト
  \item 画像の挿入
\end{itemize}

すべて正常に動作することを確認してください。

\end{document}
