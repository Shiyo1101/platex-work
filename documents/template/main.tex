\documentclass[11pt,a4paper,dvipdfmx,twocolumn]{jarticle}

% pLaTeX用パッケージ
\usepackage[dvipdfmx]{graphicx}
\usepackage[dvipdfmx]{color}
\usepackage[dvipdfmx]{hyperref}

% その他のパッケージ
\usepackage{amsmath}      % 高度な数式環境(align, equationなど)
\usepackage{amssymb}      % 数学記号の拡張(\mathbb, \mathcalなど)
\usepackage{booktabs}     % 見栄えの良い表の罫線(\toprule, \midruleなど)
\usepackage{float}        % 図表の配置制御([H]オプションでの固定配置)
\usepackage{geometry}     % ページレイアウトの設定(マージン、用紙サイズなど)
\usepackage{bookmark}     % PDFブックマークの最適化

% ドキュメントの余白、カラム同士の間隔設定
\geometry{margin=0.5in, columnsep=0.3in}

% hyperrefの設定
\hypersetup{
  pdftitle={pLaTeX 2カラムドキュメント},
  pdfauthor={著者名},
  colorlinks=true,
  linkcolor=blue,
  citecolor=blue,
  urlcolor=blue
}

% タイトル情報
\title{pLaTeX 2カラムドキュメント実践ガイド}
\author{著者名}
\date{}

\begin{document}

% タイトルをページ上部に中央配置
\twocolumn[
    \begin{@twocolumnfalse}
        \maketitle
    \end{@twocolumnfalse}
]

\section{はじめに}

本ドキュメントは,pLaTeXにおける2カラムレイアウトの実装方法を示すテンプレートです。

トップにタイトルを配置し,本文は2カラムで展開される形式となっています。このレイアウトは学会発表論文やジャーナル記事に適しています。

\section{基本的な使い方}

\subsection{段落の記述}

段落は通常通り記述できます。段落間には空行を挿入することで,自動的に改行されます。2カラムレイアウトでは行の長さが短いため,文字サイズと行間隔の調整が重要です。

本テンプレートでは,11ptのフォントサイズとマージン0.8インチを設定しており,バランスの取れた読みやすさを実現しています。

\subsection{数式の記述}

インライン数式は \(\alpha = \beta + \gamma\) のように記述します。

ディスプレイ数式は以下のように記述します:

\begin{equation}
    E = mc^2
    \label{eq:einstein}
\end{equation}

式\ref{eq:einstein}はアインシュタインの有名な方程式です。

複数行の数式を記述する場合:

\begin{align}
    f(x) & = ax^2 + bx + c                                         \\
         & = a\left(x + \frac{b}{2a}\right)^2 + c - \frac{b^2}{4a}
    \label{eq:quadratic}
\end{align}

\section{表の使い方}

\subsection{基本的な表}

表\ref{tab:example1}は基本的な表の例です。2カラムレイアウトでは,表は1カラムの幅に収まるようにコンパクトに設計すると良いでしょう。

\begin{table}[H]
    \centering
    \small
    \begin{tabular}{lcc}
        \toprule
        項目 & 値1 & 値2 \\
        \midrule
        A  & 10 & 20 \\
        B  & 30 & 40 \\
        C  & 50 & 60 \\
        \bottomrule
    \end{tabular}
    \caption{基本的な表の例}
    \label{tab:example1}
\end{table}

\subsection{複雑な表}

より複雑な表が必要な場合は,表を2段に分割するか,本文幅全体に拡張することをお勧めします。表\ref{tab:example2}では複数の行と列を示します。

\begin{table}[H]
    \centering
    \small
    \begin{tabular}{lcccr}
        \toprule
        カテゴリ & 2022年 & 2023年 & 2024年 & 変化率   \\
        \midrule
        部門A  & 100   & 120   & 145   & +45\% \\
        部門B  & 80    & 85    & 88    & +10\% \\
        部門C  & 60    & 75    & 92    & +53\% \\
        \bottomrule
    \end{tabular}
    \caption{複数年度の業績比較}
    \label{tab:example2}
\end{table}

\section{箇条書き}

\subsection{番号なしリスト}

2カラムレイアウトでのリスト表示例です:

\begin{itemize}
    \item 最初の項目は重要です
    \item 二番目の項目も同様に重要です
    \item 三番目の項目は補足情報です
    \item ネストも可能です
          \begin{itemize}
              \item サブ項目1
              \item サブ項目2
          \end{itemize}
\end{itemize}

\subsection{番号付きリスト}

段階的な手順を示す場合は番号付きリストが有効です:

\begin{enumerate}
    \item 準備段階を完了する
    \item 実装フェーズを開始する
    \item テストと検証を行う
    \item 最終調整と完成
\end{enumerate}

\section{画像の挿入}

\subsection{単一画像の挿入}

図\ref{fig:sample1}は単一の画像を挿入する例です。2カラムレイアウトでは,画像は1カラン幅に収まるサイズにする必要があります。

\begin{figure}[H]
    \centering
    \includegraphics[width=0.9\columnwidth]{images/placeholder.png}
    \caption{サンプル画像}
    \label{fig:sample1}
\end{figure}

\subsection{複数画像の並列配置}

複数の画像を並べる場合は,minipageを使用します。図\ref{fig:sample2a}と図\ref{fig:sample2b}はサイドバイサイドで表示されます。

\begin{figure}[H]
    \centering
    \begin{minipage}{0.45\columnwidth}
        \centering
        \includegraphics[width=\textwidth]{images/placeholder.png}
        \caption{左側の画像}
        \label{fig:sample2a}
    \end{minipage}
    \hfill
    \begin{minipage}{0.45\columnwidth}
        \centering
        \includegraphics[width=\textwidth]{images/placeholder.png}
        \caption{右側の画像}
        \label{fig:sample2b}
    \end{minipage}
\end{figure}

\section{参照と相互参照}

本ドキュメント内の参照を使用できます:

\begin{itemize}
    \item セクション参照:本ガイドは\ref{sec:tips}ページをご覧ください
    \item 式の参照:式\ref{eq:einstein}はノーベル賞を受賞した研究の基礎です
    \item 図の参照:図\ref{fig:sample1}に示されているように
    \item 表の参照:表\ref{tab:example1}のデータを参照
\end{itemize}

\section{2カラムレイアウトの工夫}
\label{sec:tips}

\subsection{フォントサイズの調整}

2カラムレイアウトでは,フォントサイズが重要です。本テンプレートは11ptを採用しており,10ptや12ptとの比較を推奨します。

\subsection{マージンの最適化}

\texttt{geometry}パッケージを使用して,マージン(margin)とカラム間隔(columnsep)を調整しています。デフォルト値から小さくすることで,より多くのコンテンツを配置できます。

\subsection{表と画像の最適化}

表と画像は\texttt{\string\columnwidth}を使用することで,カラム幅に自動的に適応します。

\begin{itemize}
    \item 表:\(\texttt{\string \small}\)でテキストサイズを縮小
    \item 画像:\(\texttt{width=0.9\string\columnwidth}\)で幅を指定
    \item 複数画像:\(\texttt{\string\minipage}\)で並列配置
\end{itemize}
\subsection{段落の分割}

長い段落は2カラムレイアウトで「Underfull \texttt{\string \hbox}」警告を発生させることがあります。段落を適切に分割することで改善できます。


\section{共有リソースの利用}

\subsection{共有画像ディレクトリ}

複数のドキュメントで共有する画像は,以下のように参照します:

\begin{figure}[H]
    \centering
    \includegraphics[width=0.6\columnwidth]{../../shared/images/placeholder.png}
    \caption{共有ディレクトリの画像例}
    \label{fig:shared}
\end{figure}

コード例:
\begin{verbatim}
\includegraphics{../../shared/images/
  placeholder.png}
\end{verbatim}

\subsection{カスタムスタイルファイル}

\texttt{shared/styles/}ディレクトリにカスタムスタイルがある場合:

\begin{verbatim}
\usepackage{../../shared/styles/
  mystyle}
\end{verbatim}

\subsection{参考文献の管理}

BibTeXファイルは\texttt{shared/bibliography/}に配置します。

\section{実用的なヒント}

\subsection{文字化け対策}

pLaTeXで日本語を正しく表示するには,以下の設定が必須です:

\begin{itemize}
    \item ドキュメントクラス:\(\texttt{jarticle}\)または\(\texttt{jbook}\)
    \item ファイルエンコーディング:UTF-8
    \item dvipdfmxオプション:以下を使用
\end{itemize}

\begin{verbatim}
\usepackage[dvipdfmx]{graphicx}
\end{verbatim}

\subsection{コンパイルの効率化}

latexmkコマンドを使用することで,複数回のコンパイルが自動的に実行されます:

\begin{verbatim}
latexmk -pdfdvi main.tex
\end{verbatim}

\subsection{エラートラブルシューティング}

一般的なエラーと対処法:

\begin{itemize}
    \item \textbf{ファイルが見つからない}:相対パスを確認してください
    \item \textbf{フォントエラー}:\(\texttt{jar}\)クラスを使用してください
    \item \textbf{PDFが生成されない}:DVIファイルが生成されているか確認してください
\end{itemize}

\section{結論}

本テンプレートを使用することで,専門的な2カラムドキュメントを効率的に作成できます。

必要に応じて以下をカスタマイズしてください:

\begin{itemize}
    \item タイトル,著者,日付
    \item フォントサイズとマージン
    \item セクション構成
    \item 画像と表の配置
\end{itemize}

このテンプレートが学会論文やテクニカルレポート作成の参考になれば幸いです。

\end{document}