\documentclass[12pt,a4paper]{article}

% LuaLaTeX用パッケージ
\usepackage{fontspec}
\usepackage{luatexja}
\usepackage{luatexja-fontspec}

% 日本語フォント設定(環境に応じて調整してください)
\setmainjfont{IPAMincho}
\setsansjfont{IPAGothic}

% その他のパッケージ
\usepackage{amsmath}
\usepackage{graphicx}
\usepackage{hyperref}
\usepackage{listings}
\usepackage{xcolor}

% タイトル情報
\title{LuaLaTeX サンプル文書}
\author{著者名}
\date{\today}

\begin{document}

\maketitle

\section{はじめに}

これはLuaLaTeXのサンプル文書です。日本語も問題なく表示されます。

\section{数式の例}

インライン数式の例: $E = mc^2$

ディスプレイ数式の例:
\begin{equation}
  \int_{-\infty}^{\infty} e^{-x^2} dx = \sqrt{\pi}
\end{equation}

複数行の数式:
\begin{align}
  f(x) &= x^2 + 2x + 1 \\
  &= (x + 1)^2
\end{align}

\section{リスト}

\subsection{箇条書き}

\begin{itemize}
  \item 項目1
  \item 項目2
  \item 項目3
\end{itemize}

\subsection{番号付きリスト}

\begin{enumerate}
  \item 最初の項目
  \item 二番目の項目
  \item 三番目の項目
\end{enumerate}

\section{コードの挿入}

\begin{lstlisting}[language=Python, caption=Pythonコード例]
def hello_world():
    print("Hello, World!")

if __name__ == "__main__":
    hello_world()
\end{lstlisting}

\section{まとめ}

この文書はLuaLaTeX環境のテストに使用できます。
\begin{itemize}
  \item LuaLaTeXでのコンパイル
  \item 日本語の表示
  \item 数式のレンダリング
  \item コードハイライト
\end{itemize}

すべて正常に動作することを確認してください。

\end{document}
